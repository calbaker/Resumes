%%%%%%%%%%%%%%%%%%%%%%%%%%%%%%%%%%%%%%%%%
% Plain Cover Letter
% LaTeX Template
% Version 1.0 (28/5/13)
%
% This template has been downloaded from:
% http://www.LaTeXTemplates.com
%
% Original author:
% Rensselaer Polytechnic Institute 
% http://www.rpi.edu/dept/arc/training/latex/resumes/
%
% License:
% CC BY-NC-SA 3.0 (http://creativecommons.org/licenses/by-nc-sa/3.0/)
%
%%%%%%%%%%%%%%%%%%%%%%%%%%%%%%%%%%%%%%%%%

%----------------------------------------------------------------------------------------
%	PACKAGES AND OTHER DOCUMENT CONFIGURATIONS
%----------------------------------------------------------------------------------------

\documentclass[11pt]{letter} % Default font size of the document, change to 10pt to fit more text

\usepackage{newcent} % Default font is the New Century Schoolbook PostScript font 
%\usepackage{helvet} % Uncomment this (while commenting the above line) to use the Helvetica font

% Margins
\topmargin=-1in % Moves the top of the document 1 inch above the default
\textheight=8.5in % Total height of the text on the page before text goes on to the next page, this can be increased in a longer letter
\oddsidemargin=-10pt % Position of the left margin, can be negative or positive if you want more or less room
\textwidth=6.5in % Total width of the text, increase this if the left margin was decreased and vice-versa

\let\raggedleft\raggedright % Pushes the date (at the top) to the left, comment this line to have the date on the right

\begin{document}

%----------------------------------------------------------------------------------------
%	ADDRESSEE SECTION
%----------------------------------------------------------------------------------------

\begin{letter}{}
% {Mrs. Jane Smith \\
% Recruitment Officer \\
% The Corporation \\
% 123 Pleasant Lane \\
% City, State 12345} 

%----------------------------------------------------------------------------------------
%	YOUR NAME & ADDRESS SECTION
%----------------------------------------------------------------------------------------

\begin{center}
\large\bf Chad Baker, PhD \\ % Your name
%\vspace{20pt} \hrule height 1pt % If you would like a horizontal line separating the name from the address, uncomment the line to the left of this text
calbaker@gmail.com % Your address and phone number
\end{center} 
\vfill

\signature{Chad Baker} % Your name for the signature at the bottom

%----------------------------------------------------------------------------------------
%	LETTER CONTENT SECTION
%----------------------------------------------------------------------------------------

\opening{} I have a PhD in mechanical engineering with an emphasis in
thermal/fluid systems from the University of Texas at Austin.  I have
spent the past 6 years working as a Thermal Systems Research Engineer
at Ford motor company, developing thermal models that enable vehicle
level fuel economy simulations.  In addition, I've developed an
extensible object-oriented Modelica package for the modeling of
vehicle thermal systems and components (as well as some electrical and
mechanical domain components) that is used by around 30 engineers in
Ford Motor Company for both analysis and new vehicle model
development.  I've also been instrumental in developing a process for
utilizing IBM Rational Team Concert version control and workflow
management software, and I've been heavily involved in training others
to use the software and politely enforcing process discipline.  Other
work I've done at Ford has included multi-objective Pareto
optimization of semi-empirical, physics-based models and
phenomenological meta models using ModeFRONTIER and other tools.  I've
modeled in thermal, electrical, and mechanical domains.  I've
implemented hybrid and battery electric vehicle battery models for
fuel economy and range simulations. All of my work at Ford has been
directed at reducing fossil fuel consumption by improving fuel economy
in hybrid electric vehicles and range in battery electric vehicles.
Prior to that, I had extensive experience working with Python,
MATLAB/Simulink (which I still use daily at Ford), Git version
control, Linux, National Instruments products, and various other
computational tools.

When I was in the process of finishing graduate school, my goal was to
spend my career doing the most I could to reduce humanity's dependence
on fossil fuels because I view this as an important technical
challenge.  Now that the auto industry is getting further along toward
full electrification, I think the technical hurdles in the way of
addressing that challenge (of reducing dependence on fossil fuels) are
moving.  I think now the key issues that need to be addressed are in
the production, distribution, and storage of electricity to create the
infrastructure needed for electrified transportation and other end
uses.  I think nuclear energy should a key part of this, and I think
my skill set could be well applied to the thermal challenges
associated with nuclear energy.

In the application process, I filled out June 1, 2019 as my
anticipated last day at Ford.  That is when I will become available
for work at Oak Ridge National Lab.

\closing{Sincerely,}


\encl{resume} % List your enclosed documents here, comment this out to get rid of the "encl:"

%----------------------------------------------------------------------------------------

\end{letter}

\end{document}