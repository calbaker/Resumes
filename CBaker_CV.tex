% LaTeX resume using res.cls
\documentclass[centered]{res} 
\usepackage{helvetica} % uses helvetica postscript font (download helvetica.sty)
%\usepackage{newcent}   % uses new century schoolbook postscript font 
\usepackage{color,hyperref}
\definecolor{darkblue}{rgb}{0.0,0.0,0.3}
\hypersetup{colorlinks,breaklinks,
            linkcolor=darkblue,urlcolor=darkblue,
            anchorcolor=darkblue,citecolor=darkblue}
\hyphenpenalty=100000
\sectionskip=8pt

\begin{document}

% Center the name over the entire width of resume:
 \moveleft.5\hoffset\centerline{\large\bf Chad A. Baker}
% Draw a horizontal line the whole width of resume:
 \moveleft\hoffset\vbox{\hrule width\resumewidth height 1pt}\smallskip
% address begins here
% Again, the address lines must be centered over entire width of resume:
 \moveleft.5\hoffset\centerline{calbaker@utexas.edu}
 \moveleft.5\hoffset\centerline{214-695-4636}
 
\begin{resume}
 
% Notes for myself (CAB).  For a CV move Presentations and
% Publications ahead of professional experience.  Add sections for
% teaching experience, teaching interests, research experience, and
% research interests.  Also, add a section for the dissertation
% abstract.  Also, add section for leadership experience.  

\section{Objective}

Post-doc or career position utilizing skills in modeling and
experimentation to solve challenging technical problems that address
energy or environmental problems facing humanity.

\section{Research Interests}
% maybe make this a list environment
Modeling of mass, heat, and energy transport processes; combustion
modeling; catalyst modeling; heat exchanger optimization; validating
models through experimentation
 
\section{Education}
%
\href{http://www.utexas.edu/}{\textbf{The University of Texas at
    Austin}}, Austin, TX USA \\
Ph.D., \href{http://www.me.utexas.edu/}{Mechanical Engineering},
August 2012 \\
Advisers:
\href{http://www.me.utexas.edu/directory/faculty/hall/matthew/33/}{Professor
  Matthew J. Hall} and \href{http://www.me.utexas.edu/~lishi/}{Professor Li Shi}\\
Area of Study: Thermal/Fluids Systems \\
GPA: 3.67/4.00 \vspace{5pt} \\
%
\href{http://www.utexas.edu/}{\textbf{The University of Texas at
    Austin}}, Austin, TX USA \\
Master of Science, \href{http://www.me.utexas.edu/}{Mechanical
  Engineering}, August
2009 \\
Thesis Topic: \emph{Vapor Transport Techniques for Growing
  Macroscopically Uniform Zinc Oxide Nanowires} \\
Advisers:
\href{http://www.me.utexas.edu/directory/faculty/hall/matthew/33/}{Professor
  Matthew J. Hall} and \href{http://www.me.utexas.edu/~lishi/}{Professor Li Shi}\\
Area of Study: Thermal/Fluids Systems \\
GPA: 3.46/4.00 
%\vspace{5pt} \\

\href{http://www.tamu.edu/}{\textbf{Texas A\&M University}}, College
Station, TX USA \\
Bachelor of Science, \href{http://www.mengr.tamu.edu/}{Mechanical
  Engineering}, May
2007  \\
GPA: 3.23/4.00

\section{Professional Experience}

\textit{Graduate Research Assistant} \\
\href{http://www.utexas.edu/}{\textbf{The University of Texas at
    Austin}}, Austin, TX USA%
\hfill \textbf{October 2010 to Present} \\
Thermoelectric diesel exhaust waste heat recovery to improve fuel
efficiency
\begin{itemize}
\item Developed two dimensional thermoelectric device model with
  iterative coupling to heat exchanger model
\item Designed heat exchanger experiment to test thermoelectric
  devices in Cummins engine exhaust
\end{itemize}
%
\textit{Graduate Research Assistant} \\
\href{http://www.utexas.edu/}{\textbf{The University of Texas at
    Austin}}, Austin, TX USA%
\hfill \textbf{January 2008 to Present} \\
ZnO nanowires as novel catalyst substrate with analytical species
transport model
\begin{itemize} \itemsep -2pt % reduce space between items
\item Developed two dimensional analytical model for catalyst
  species concentration
\item Developed scalable method for chemical vapor transport ZnO
  nanowire growth
\item Designed heterogeneous combustion reactor for testing species
  transport enhancement caused by ZnO nanowires
\end{itemize} 
%
\textit{Graduate Technical Intern} \\
\href{http://www.lyondellbasell.com/LandingPages/SolvayEngineeredPolymers}{\textbf{Solvay
    Engineered Polymers}}, Mansfield, TX USA 
\hfill \textbf{May 2007 to August 2007} \\
Polymer additives to increase durability of raw material for Ford King
Ranch F-150 wheel flare
\begin{itemize} \itemsep -2pt % reduce space between items
\item Designed experiments to test additive migration
  and degradation in thermoplastics.
\item Trained researchers to use lab equipment such as tensile
  testers and injection molders.
\item Assisted other interns in Project Engineering with
  reducing production down time.
\end{itemize}
%
\newpage{} 
%
\textit{Project Engineering Intern} \\
\href{http://www.lyondellbasell.com/LandingPages/SolvayEngineeredPolymers}{\textbf{Solvay
    Engineered Polymers}}, Mansfield, TX USA 
\hfill \textbf{May 2006 to August 2006} \\
Polymer manufacturing plant improvements 
\begin{itemize} \itemsep -2pt % reduce space between items
\item Successfully designed improved cooling system for 3000
  horsepower motors to prevent motor failures.
\item Managed contractors for installation of improved lighting,
  safety equipment, and various structures to improve serviceability
  of polymer production lines.
\end{itemize}
%
\textit{Machinist} \\
\textbf{Texas A\&M ECAE Lab}, College Station, TX USA
\hfill \textbf{June 2005 to August 2005} \\
Machining in support of metallurgical research 
\begin{itemize} \itemsep -2pt % reduce space between items
\item Planned and implemented machining processes to produce test
  billets from a raw round ingot.
\item Assisted with construction of a custom annealing furnace.
\item Assisted with operation of equal channel angular extruder.
\end{itemize}

\section{Academic Experience}
\label{sec:teaching}

\textit{Grader} \\
\href{http://www.utexas.edu/}{\textbf{The University of Texas at
    Austin}}, Austin, TX USA
\hfill \textbf{September 2001 to December 2001} \\
Graded weekly assignments for ME 374C - Internal Combustion Engines 

\textit{Teaching Assistant} \\
\href{http://www.utexas.edu/}{\textbf{The University of Texas at
    Austin}}, Austin, TX USA
\hfill \textbf{September 2007 to December 2007} \\
Teaching assistant for ME 139L - Heat Transfer Lab
\begin{itemize} \itemsep -2pt
\item Led lab sessions 
\item Graded lab reports, quizzes, and tests
\item Provided instruction on effective technical writing
\end{itemize}

\section{Leadership and Service}
% GAIN
\textit{Co-Director} \\
\href{http://gain.engr.utexas.edu/}{Graduate and Industry Networking},
\href{http://www.utexas.edu}{The University of Texas at Austin}
\hfill \textbf{May 2009 to May 2010}
\begin{itemize} \itemsep -2pt % reduce space between items
\item Established communication with engineers in industry to attend the networking event and provide funding.
\item Secured funding for awards for poster session and paper competition.
\item Secured facility for hosting the event.  
\end{itemize}

% GEC
\textit{Academic Career Seminar Series Director} \\
\href{http://sites.google.com/site/utexasgecouncil/}{Graduate Engineering Council (GEC)},
\href{http://www.utexas.edu}{The University of Texas at Austin},
\hfill \textbf{May 2011 to May 2012} \\ 
Organized seminar series for preparing students for academic jobs

\textit{President} \\
\href{http://sites.google.com/site/utexasgecouncil/}{Graduate Engineering Council (GEC)},
\href{http://www.utexas.edu}{The University of Texas at Austin},
\hfill \textbf{Spring 2011}
\begin{itemize} \itemsep -2pt % reduce space between items
\item Interfaced with university level student government
\item Assisted other officers with their GEC projects
\item Voiced concerns of GEC to Engineering Assistant Dean of Academic
  Affairs 
\end{itemize}

\textit{Vice President} \\
\href{http://sites.google.com/site/utexasgecouncil/}{Graduate Engineering Council (GEC)},
\href{http://www.utexas.edu}{The University of Texas at Austin},
  \hfill \textbf{Fall 2010}
  \begin{itemize} \itemsep -2pt % reduce space between items
  \item Directed all GEC internal activities
  \item Assisted other officers with their GEC projects
  \item Rewrote student organization constitution to reflect GEC's actual duties
  \end{itemize}~

\newpage{}General Service
\begin{itemize} \itemsep -2pt % reduce space between items
\item Refereed for First Lego League competitions to encourage
  children to be interested in science and engineering
\item Demonstrated thermoelectric waste heat recovery research at
  Introduce a Girl to Engineering Day
\item Demonstrated thermoelectric waste heat recovery research at
  Explore UT 
\end{itemize}

\section{Publications}

Baker, C., P. Vuppuluri, M. Hall, L. Shi., \textit{Model of Heat
  Exchanger for Waste Heat Recovery from Diesel Engine Exhaust for
  Thermoelectric Power Generation}, Submitted for peer-reviewed
publication in a special issue of the Journal of Electronic Materials
for \href{http://ict2011.its.org/}{ICT 2011} conference proceedings.

\section{Presentations}
C. Baker, invited presentation: \textit{The Importance of System Level
Parameter Optimization in Thermoelectric Waste Heat Recovery,}
\href{http://www.micropower-global.com/our_company/texasstate/}{MicroPower},
October 2011. \\
C. Baker, P. Vuppuluri, L. Shi, M. Hall, \textit{Model of Heat Exchanger for
Recovering Waste Heat from Diesel Engine Exhaust for Thermoelectric
Power Generation,}
\href{http://ict2011.its.org/}{ICT 2011}, July 2011. \\
C. Baker, A. Osman Emiroglu, M. Hall, L. Shi, \textit{ZnO Nanowires as a
Novel Catalyst Substrate}, \href{http://www.22nam.org/}{22nd North
  American Catalysis Society Meeting}, June 2011.

\section{Technical Skills}
% 
Instrumentation and Control:
\href{http://www.mathworks.com/products/simulink/}{Simulink},
\href{http://www.ni.com/}{LabVIEW}, 
\href{http://www.omega.com/}{Omega} temperature controllers,
and
\href{http://www.ni.com}{National Instruments}
control and data acquisition hardware and software
\vspace{5pt} \\
%
Applications: Microsoft Office, Python,
\href{http://www.mathworks.com/products/matlab/}{\textsc{Matlab}},
  \LaTeX, LyX, Emacs, and other common productivity packages for
  Windows and Linux platforms
\vspace{5pt} \\
%
Python/\href{http://www.scipy.org/}{SciPy}/%
\href{http://numpy.scipy.org/}{NumPy}
experience: linear algebra, polynomials, optimization, curve fitting,
object oriented programming, emission control catalyst modeling, heat
exchanger modeling, thermoelectric device modeling
\vspace{5pt} \\
%
\href{http://www.mathworks.com/products/matlab/}{\textsc{Matlab}}
experience: linear algebra, polynomials, stiff ODEs, vehicle dynamics
modeling, combustion catalyst modeling, vapor deposition modeling
\vspace{5pt} \\
%
Operating Systems: MS Windows XP, MS Windows 7, and Ubuntu Linux
\vspace{5pt}
%
\section{Awards}
% 
\href{http://www.utexas.edu/}{\textbf{The University of Texas at
    Austin}}, Austin, TX USA \\
Richard J. Kokes Travel Award for
\href{http://www.22nam.org/}{22nd North American Catalysis Society
  Meeting} \\
Association of Energy Engineers Foundation Scholarship Award \\
N.K. Wright Centennial Memorial Endowed Presidential Scholarship \\
Nano Night '08 Best Poster \vspace{5pt} \\
%
\href{http://www.tamu.edu/}{\textbf{Texas A\&M University}}, College
Station, TX USA \\
American Society of Materials (ASM) International (Houston Chapter)
Scholarship \\
Society of Fire Protection Engineers Scholarship \vspace{5pt} \\
%
\textbf{Prior to University Education} \\
Eagle Scout

\end{resume}
\end{document}







