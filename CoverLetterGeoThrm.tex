%%%%%%%%%%%%%%%%%%%%%%%%%%%%%%%%%%%%%%%%%
% Plain Cover Letter
% LaTeX Template
% Version 1.0 (28/5/13)
%
% This template has been downloaded from:
% http://www.LaTeXTemplates.com
%
% Original author:
% Rensselaer Polytechnic Institute 
% http://www.rpi.edu/dept/arc/training/latex/resumes/
%
% License:
% CC BY-NC-SA 3.0 (http://creativecommons.org/licenses/by-nc-sa/3.0/)
%
%%%%%%%%%%%%%%%%%%%%%%%%%%%%%%%%%%%%%%%%%

%----------------------------------------------------------------------------------------
%	PACKAGES AND OTHER DOCUMENT CONFIGURATIONS
%----------------------------------------------------------------------------------------

\documentclass[11pt]{letter} % Default font size of the document, change to 10pt to fit more text

\usepackage{newcent} % Default font is the New Century Schoolbook PostScript font 
%\usepackage{helvet} % Uncomment this (while commenting the above line) to use the Helvetica font

% Margins
\topmargin=-1in % Moves the top of the document 1 inch above the default
\textheight=8.5in % Total height of the text on the page before text goes on to the next page, this can be increased in a longer letter
\oddsidemargin=-10pt % Position of the left margin, can be negative or positive if you want more or less room
\textwidth=6.5in % Total width of the text, increase this if the left margin was decreased and vice-versa

\let\raggedleft\raggedright % Pushes the date (at the top) to the left, comment this line to have the date on the right

\begin{document}

%----------------------------------------------------------------------------------------
%	ADDRESSEE SECTION
%----------------------------------------------------------------------------------------

\begin{letter}{}
% {Mrs. Jane Smith \\
% Recruitment Officer \\
% The Corporation \\
% 123 Pleasant Lane \\
% City, State 12345} 

%----------------------------------------------------------------------------------------
%	YOUR NAME & ADDRESS SECTION
%----------------------------------------------------------------------------------------

\begin{center}
\large\bf Chad Baker, PhD \\ % Your name
%\vspace{20pt} \hrule height 1pt % If you would like a horizontal line separating the name from the address, uncomment the line to the left of this text
calbaker@gmail.com % Your address and phone number
\end{center} 
\vfill

\signature{Chad Baker} % Your name for the signature at the bottom

%----------------------------------------------------------------------------------------
%	LETTER CONTENT SECTION
%----------------------------------------------------------------------------------------

\opening{} I have a PhD in mechanical engineering with an emphasis in
thermal/fluid systems from the University of Texas at Austin.  I have
spent the past 6 years working as a Thermal Systems Research Engineer
at Ford motor company, developing thermal models that enable vehicle
level fuel economy simulations.  In addition, I've developed an
extensible object-oriented Modelica package for the modeling of
vehicle thermal systems and components (as well as some electrical and
mechanical domain components) that is used by around 30 engineers in
Ford Motor Company for both analysis and new vehicle model
development.  I've also been instrumental in developing a process for
utilizing IBM Rational Team Concert version control and workflow
management software, and I've been heavily involved in training others
to use the software and politely enforcing process discipline.  Other
work I've done at Ford has included multi-objective Pareto
optimization of semi-empirical, physics-based models and
phenomenological meta models using ModeFRONTIER and other tools.  All
of my work at Ford has been directed at reducing fossil fuel
consumption by improving fuel economy in hybrid electric vehicles and
range in battery electric vehicles.  Prior to that, I had extensive
experience working with Python, MATLAB/Simulink (which I still use
daily at Ford), Git version control, Linux, National Instruments
products, and various other computational tools.

When I was in the process of finishing graduate school, my goal was to
spend my career doing the most I could to reduce humanity's dependence
on fossil fuels.  I view this as the most crucial technical challenge
that will face us in my lifetime.  Now that the auto industry is
getting further along toward full electrification, I think the
technical hurdles in the way of addressing that challenge (of reducing
dependence on fossil fuels) are moving.  I think now the key issues
that need to be addressed are in the production, distribution, and
storage of electricity, as well as demand side reduction in private
and commercial buildings.

Last September, I went to Iceland for vacation, and I learned that
their power grid is entirely geothermal!  I toured a powerplant there,
and I was amazed to see how meticulously they designed the processes
for extracting enthalpy from the steam to maximize efficiency and
minimize water usage.  I learned that they do this because geothermal
energy is not, in fact, limitless.  Iceland has apparently discovered
that even in such a wet environment, the ground water can be depleted,
and the thermal energy can be excessively extracted in the vicinity of
the geothermal well.  As such, they carefully analyze the potential of
each geothermal site to ensure that it can be operated sustainably.  I
also learned that the American west has a largely untapped potential
for geothermal power generation, which could greatly improve our
energy independence, sustainability, reliability, and general
robustness if we're careful to do it right in terms of managing the
energy extraction and water usage (for example, by using a closed flow
loop rather than an open system).  I would like to be a part of this
mission because I believe my experience and talents could be of
particular value, and because it is deeply interesting to me, I would
apply a great deal of passion toward this effort.

Please consider me for this position.  

\closing{Sincerely,}


\encl{resume} % List your enclosed documents here, comment this out to get rid of the "encl:"

%----------------------------------------------------------------------------------------

\end{letter}

\end{document}