%%%%%%%%%%%%%%%%%%%%%%%%%%%%%%%%%%%%%%%%%%%%%%%%%%%%%%%%%%%%%%%%%%%%%%%%
%%%%%%%%%%%%%%%%%%%%%% Simple LaTeX CV Template %%%%%%%%%%%%%%%%%%%%%%%%
%%%%%%%%%%%%%%%%%%%%%%%%%%%%%%%%%%%%%%%%%%%%%%%%%%%%%%%%%%%%%%%%%%%%%%%%

%%%%%%%%%%%%%%%%%%%%%%%%%%%%%%%%%%%%%%%%%%%%%%%%%%%%%%%%%%%%%%%%%%%%%%%%
%% NOTE: If you find that it says                                     %%
%%                                                                    %%
%%                           1 of ??                                  %%
%%                                                                    %%
%% at the bottom of your first page, this means that the AUX file     %%
%% was not available when you ran LaTeX on this source. Simply RERUN  %%
%% LaTeX to get the ``??'' replaced with the number of the last page  %%
%% of the document. The AUX file will be generated on the first run   %%
%% of LaTeX and used on the second run to fill in all of the          %%
%% references.                                                        %%
%%%%%%%%%%%%%%%%%%%%%%%%%%%%%%%%%%%%%%%%%%%%%%%%%%%%%%%%%%%%%%%%%%%%%%%%

%%%%%%%%%%%%%%%%%%%%%%%%%%%% Document Setup %%%%%%%%%%%%%%%%%%%%%%%%%%%%

% Don't like 10pt? Try 11pt or 12pt
\documentclass[10pt]{article}

% This is a helpful package that puts math inside length specifications
\usepackage{calc}

% Simpler bibsection for CV sections
% (thanks to natbib for inspiration)
\makeatletter
\newlength{\bibhang}
\setlength{\bibhang}{1em}
\newlength{\bibsep}
 {\@listi \global\bibsep\itemsep \global\advance\bibsep by\parsep}
\newenvironment{bibsection}
    {\minipage[t]{\linewidth}\list{}{%
        \setlength{\leftmargin}{\bibhang}%
        \setlength{\itemindent}{-\leftmargin}%
        \setlength{\itemsep}{\bibsep}%
        \setlength{\parsep}{\z@}%
        }}
    {\endlist\endminipage}
\makeatother

% Layout: Puts the section titles on left side of page
\reversemarginpar

%
%         PAPER SIZE, PAGE NUMBER, AND DOCUMENT LAYOUT NOTES:
%
% The next \usepackage line changes the layout for CV style section
% headings as marginal notes. It also sets up the paper size as either
% letter or A4. By default, letter was used. If A4 paper is desired,
% comment out the letterpaper lines and uncomment the a4paper lines.
%
% As you can see, the margin widths and section title widths can be
% easily adjusted.
%
% ALSO: Notice that the includefoot option can be commented OUT in order
% to put the PAGE NUMBER *IN* the bottom margin. This will make the
% effective text area larger.
%
% IF YOU WISH TO REMOVE THE ``of LASTPAGE'' next to each page number,
% see the note about the +LP and -LP lines below. Comment out the +LP
% and uncomment the -LP.
%
% IF YOU WISH TO REMOVE PAGE NUMBERS, be sure that the includefoot line
% is uncommented and ALSO uncomment the \pagestyle{empty} a few lines
% below.
%

%% Use these lines for letter-sized paper
\usepackage[paper=letterpaper,
            %includefoot, % Uncomment to put page number above margin
            marginparwidth=1.2in,     % Length of section titles
            marginparsep=.05in,       % Space between titles and text
            margin=1in,               % 1 inch margins
            includemp]{geometry}

%% Use these lines for A4-sized paper
%\usepackage[paper=a4paper,
%            %includefoot, % Uncomment to put page number above margin
%            marginparwidth=30.5mm,    % Length of section titles
%            marginparsep=1.5mm,       % Space between titles and text
%            margin=25mm,              % 25mm margins
%            includemp]{geometry}

%% More layout: Get rid of indenting throughout entire document
\setlength{\parindent}{0in}

%% This gives us fun enumeration environments. compactitem will be nice.
\usepackage{paralist}

%% Reference the last page in the page number
%
% NOTE: comment the +LP line and uncomment the -LP line to have page
%       numbers without the ``of ##'' last page reference)
%
% NOTE: uncomment the \pagestyle{empty} line to get rid of all page
%       numbers (make sure includefoot is commented out above)
%
\usepackage{fancyhdr,lastpage}
\pagestyle{fancy}
%\pagestyle{empty}      % Uncomment this to get rid of page numbers
\fancyhf{}\renewcommand{\headrulewidth}{0pt}
\fancyfootoffset{\marginparsep+\marginparwidth}
\newlength{\footpageshift}
\setlength{\footpageshift}
          {0.5\textwidth+0.5\marginparsep+0.5\marginparwidth-2in}
\lfoot{\hspace{\footpageshift}%
       \parbox{4in}{\, \hfill %
                    \arabic{page} of \protect\pageref*{LastPage} % +LP
%                    \arabic{page}                               % -LP
                    \hfill \,}}

% Finally, give us PDF bookmarks
\usepackage{color,hyperref}
\definecolor{darkblue}{rgb}{0.0,0.0,0.3}
\hypersetup{colorlinks,breaklinks,
            linkcolor=darkblue,urlcolor=darkblue,
            anchorcolor=darkblue,citecolor=darkblue}

%%%%%%%%%%%%%%%%%%%%%%%% End Document Setup %%%%%%%%%%%%%%%%%%%%%%%%%%%%


%%%%%%%%%%%%%%%%%%%%%%%%%%% Helper Commands %%%%%%%%%%%%%%%%%%%%%%%%%%%%

% The title (name) with a horizontal rule under it
%
% Usage: \makeheading{name}
%
% Place at top of document. It should be the first thing.
\newcommand{\makeheading}[1]%
        {\hspace*{-\marginparsep minus \marginparwidth}%
         \begin{minipage}[t]{\textwidth+\marginparwidth+\marginparsep}%
                {\large \bfseries #1}\\[-0.15\baselineskip]%
                 \rule{\columnwidth}{1pt}%
         \end{minipage}}

% The section headings
%
% Usage: \section{section name}
%
% Follow this section IMMEDIATELY with the first line of the section
% text. Do not put whitespace in between. That is, do this:
%
%       \section{My Information}
%       Here is my information.
%
% and NOT this:
%
%       \section{My Information}
%
%       Here is my information.
%
% Otherwise the top of the section header will not line up with the top
% of the section. Of course, using a single comment character (%) on
% empty lines allows for the function of the first example with the
% readability of the second example.
\renewcommand{\section}[2]%
        {\pagebreak[2]\vspace{1.3\baselineskip}%
         \phantomsection\addcontentsline{toc}{section}{#1}%
         \hspace{0in}%
         \marginpar{
         \raggedright \scshape #1}#2}

% An itemize-style list with lots of space between items
\newenvironment{outerlist}[1][\enskip\textbullet]%
        {\begin{itemize}[#1]}{\end{itemize}%
         \vspace{-.6\baselineskip}}

% An environment IDENTICAL to outerlist that has better pre-list spacing
% when used as the first thing in a \section
\newenvironment{lonelist}[1][\enskip\textbullet]%
        {\vspace{-\baselineskip}\begin{list}{#1}{%
        \setlength{\partopsep}{0pt}%
        \setlength{\topsep}{0pt}}}
        {\end{list}\vspace{-.6\baselineskip}}

% An itemize-style list with little space between items
\newenvironment{innerlist}[1][\enskip\textbullet]%
        {\begin{compactitem}[#1]}{\end{compactitem}}

% To add some paragraph space between lines.
% This also tells LaTeX to preferably break a page on one of these gaps
% if there is a needed pagebreak nearby.
\newcommand{\blankline}{\quad\pagebreak[2]}

% 

%%%%%%%%%%%%%%%%%%%%%%%% End Helper Commands %%%%%%%%%%%%%%%%%%%%%%%%%%%

%%%%%%%%%%%%%%%%%%%%%%%%% Begin CV Document %%%%%%%%%%%%%%%%%%%%%%%%%%%%

\begin{document}
\makeheading{Chad A. Baker}

\section{Contact Information}
%
% NOTE: Mind where the & separators and \\ breaks are in the following
%       table.
%
% ALSO: \rcollength is the width of the right column of the table
%       (adjust it to your liking; default is 1.85in).
%
\newlength{\rcollength}\setlength{\rcollength}{1.85in}%
% The next command inserts a table.  Use '&' to separate elements of the table.
\begin{tabular}[t]{@{}p{\textwidth-\rcollength}p{\rcollength}}
\href{http://www.me.utexas.edu//}%
     {Department of Mechanical Engineering} & \\
\href{http://www.utexas.edu/}{The University of Texas at Austin} 
									& \textit{Cell:} (214) 695-4636 \\
1 University Station, C2200 \\
Austin, TX 78712-0292 \\
\textit{E-mail:} \href{mailto:calbaker@utexas.edu}{calbaker@mail.utexas.edu}
& \textit{E-mail:} \href{mailto:calbaker@gmail.com}{calbaker@gmail.com}\\
\end{tabular}

%\section{Citizenship}
%%
%USA

\section{Interests}
%
offering expert advice on technology investment; modeling of mass,
heat, and energy transport processes; combustion modeling; catalyst
optimization; heat exchanger optimization; second law efficiency
optimization

\section{Education}
%
\href{http://www.utexas.edu/}{\textbf{The University of Texas at Austin}}, Austin, TX USA
\begin{outerlist}
\item[] Ph.D.,
        \href{http://www.me.utexas.edu/}
             {Mechanical Engineering}, Expected Graduation: Dec 2012
        \begin{innerlist}        
        \item Adviser:%
              \href{http://www.me.utexas.edu/directory/faculty/hall/matthew/33/}
                   {Professor Matthew J. Hall}
        \item Area of Study: Thermal/Fluids Systems
        \item GPA: 3.67/4.00
        \end{innerlist}
\end{outerlist}

\blankline

\href{http://www.utexas.edu/}{\textbf{The University of Texas at Austin}}, Austin, TX USA
\begin{outerlist}
\item[] M.S.,
        \href{http://www.me.utexas.edu/}
             {Mechanical Engineering}, August 2009
        \begin{innerlist}
        \item Thesis Topic: \emph{Vapor Transport Techniques for
            Growing Macroscopically Uniform Zinc Oxide Nanowires} 
        \item Adviser:%
              \href{http://www.me.utexas.edu/directory/faculty/hall/matthew/33/}
                   {Professor Matthew J. Hall}
        \item Area of Study: Thermal/Fluids Systems
        \item GPA: 3.46/4.00
        \end{innerlist}
\end{outerlist}

\blankline

\href{http://www.tamu.edu/}{\textbf{Texas A\&M University}}, College Station, TX USA
\begin{outerlist}
\item[] B.S.,
        \href{http://www.mengr.tamu.edu/}{Mechanical Engineering}, May
        2007 
        \begin{innerlist}
        \item GPA: 3.23/4.00
        \end{innerlist}
\end{outerlist}

\section{Professional Experience}
%
\href{http://www.lyondellbasell.com/LandingPages/SolvayEngineeredPolymers}{\textbf{Solvay Engineered Polymers}}, Mansfield, TX USA
\begin{outerlist}

\item[] \textit{Graduate Technical Intern}%
        \hfill \textbf{May 2007 to August 2007}
\begin{innerlist}
\item Designed experiments to test additive migration and degradation in thermoplastics.
\item Trained researchers to use lab equipment such as tensile testers and injection molders.
\item Assisted other interns in Project Engineering with reducing production down time.
\end{innerlist}

\item[] \textit{Project Engineering Intern}%
        \hfill \textbf{May 2006 to August 2006}
\begin{innerlist}
\item Successfully designed improved cooling system for 3000 horsepower motors to prevent motor failures.
\item Managed contractors for installation of improved lighting, safety equipment, and various structures to improve serviceability of polymer production lines.  
\end{innerlist}

\end{outerlist}

\blankline

\href{http://msen.tamu.edu/labfacilities.html}{\textbf{Texas A\&M ECAE Lab}}, College Station, TX USA
\begin{outerlist}
\item[] \textit{Machinist}%
        \hfill \textbf{June 2005 to August 2005}
\begin{innerlist}
\item Planned and implemented machining processes to produce test
  billets from a raw round ingot.
\item Assisted with construction of a custom annealing furnace.
\item Assisted with operation of equal channel angular extruder.
\end{innerlist}

\end{outerlist}

\section{Technical Skills}
%
Instrumentation and Control:
        \href{http://www.mathworks.com/products/simulink/}{Simulink},
        \href{http://www.ni.com/}{LabVIEW}, 
        \href{http://www.omega.com/}{Omega} temperature controllers,
        and
        \href{http://www.ni.com}{National Instruments}
        control and data acquisition hardware and software

\blankline

Applications: Microsoft Office, Python,
\href{http://www.mathworks.com/products/matlab/}{Matlab}, \LaTeX, LyX,
Emacs, and other common productivity packages for Windows and Linux
platforms

\blankline

Python/\href{http://www.scipy.org/}{SciPy}/%
\href{http://numpy.scipy.org/}{NumPy}
experience: linear algebra, polynomials, optimization, curve fitting,
object oriented programming, emission control catalyst modeling, heat
exchanger modeling, thermoelectric device modeling

\blankline

\href{http://www.mathworks.com/products/matlab/}{\textsc{Matlab}}
experience: linear algebra, polynomials, stiff ODEs, vehicle dynamics
modeling, combustion catalyst modeling, vapor deposition modeling

\blankline


Operating Systems: MS Windows XP, MS Windows 7, and Ubuntu Linux

\section{Awards}
%
\href{http://www.utexas.edu/}{\textbf{The University of Texas at Austin}}, Austin, TX USA
\begin{innerlist}
\item Richard J. Kokes Travel Award
\item Association of Energy Engineers Foundation Scholarship Award
\item N.K. Wright Centennial Memorial Endowed Presidential Scholarship
\item Nano Night '08 Best Poster
\end{innerlist}

\blankline

\href{http://www.tamu.edu/}{\textbf{Texas A\&M University}}, College Station, TX USA
\begin{innerlist}
\item American Society of Materials (ASM) International (Houston Chapter) Scholarship
\item Society of Fire Protection Engineers Scholarship
\end{innerlist}

\blankline

\textbf{Prior to University Education}
\begin{innerlist}
\item Eagle Scout
\end{innerlist}

\section{Academic Experience}
%
\href{http://www.utexas.edu/}{\textbf{The University of Texas at Austin}}, Austin, TX USA
\begin{outerlist}

\item[] \textit{Graduate Research Assistant}%
    \hfill \textbf{January 2008 to Present}
	    \begin{innerlist}
        \item Developed scalable method for chemical vapor transport ZnO nanowire growth
        \item Designed heterogeneous combustion reactor for testing species transport enhancement
        caused by ZnO nanowires
        \end{innerlist}~

\item[] \textit{Teaching Assistant}%
        \hfill \textbf{September 2007 to December 2007}
\begin{innerlist}
	\item Led lab sessions for ME 139L - Heat Transfer Lab
	\item Graded lab reports, quizzes, and tests
	\item Provided instruction on effective technical writing
\end{innerlist}~
\end{outerlist}

\end{document}

%%%%%%%%%%%%%%%%%%%%%%%%%% End CV Document %%%%%%%%%%%%%%%%%%%%%%%%%%%%%
